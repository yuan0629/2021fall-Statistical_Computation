\documentclass[11pt]{article}

    \usepackage[breakable]{tcolorbox}
    \usepackage{parskip} % Stop auto-indenting (to mimic markdown behaviour)
    \author{Congyuan Duan}
    \usepackage{iftex}
    \ifPDFTeX
    	\usepackage[T1]{fontenc}
    	\usepackage{mathpazo}
    \else
    	\usepackage{fontspec}
    \fi

    % Basic figure setup, for now with no caption control since it's done
    % automatically by Pandoc (which extracts ![](path) syntax from Markdown).
    \usepackage{graphicx}
    % Maintain compatibility with old templates. Remove in nbconvert 6.0
    \let\Oldincludegraphics\includegraphics
    % Ensure that by default, figures have no caption (until we provide a
    % proper Figure object with a Caption API and a way to capture that
    % in the conversion process - todo).
    \usepackage{caption}
    \DeclareCaptionFormat{nocaption}{}
    \captionsetup{format=nocaption,aboveskip=0pt,belowskip=0pt}

    \usepackage[Export]{adjustbox} % Used to constrain images to a maximum size
    \adjustboxset{max size={0.9\linewidth}{0.9\paperheight}}
    \usepackage{float}
    \floatplacement{figure}{H} % forces figures to be placed at the correct location
    \usepackage{xcolor} % Allow colors to be defined
    \usepackage{enumerate} % Needed for markdown enumerations to work
    \usepackage{geometry} % Used to adjust the document margins
    \usepackage{amsmath} % Equations
    \usepackage{amssymb} % Equations
    \usepackage{textcomp} % defines textquotesingle
    % Hack from http://tex.stackexchange.com/a/47451/13684:
    \AtBeginDocument{%
        \def\PYZsq{\textquotesingle}% Upright quotes in Pygmentized code
    }
    \usepackage{upquote} % Upright quotes for verbatim code
    \usepackage{eurosym} % defines \euro
    \usepackage[mathletters]{ucs} % Extended unicode (utf-8) support
    \usepackage{fancyvrb} % verbatim replacement that allows latex
    \usepackage{grffile} % extends the file name processing of package graphics 
                         % to support a larger range
    \makeatletter % fix for grffile with XeLaTeX
    \def\Gread@@xetex#1{%
      \IfFileExists{"\Gin@base".bb}%
      {\Gread@eps{\Gin@base.bb}}%
      {\Gread@@xetex@aux#1}%
    }
    \makeatother

    % The hyperref package gives us a pdf with properly built
    % internal navigation ('pdf bookmarks' for the table of contents,
    % internal cross-reference links, web links for URLs, etc.)
    \usepackage{hyperref}
    % The default LaTeX title has an obnoxious amount of whitespace. By default,
    % titling removes some of it. It also provides customization options.
    \usepackage{titling}
    \usepackage{longtable} % longtable support required by pandoc >1.10
    \usepackage{booktabs}  % table support for pandoc > 1.12.2
    \usepackage[inline]{enumitem} % IRkernel/repr support (it uses the enumerate* environment)
    \usepackage[normalem]{ulem} % ulem is needed to support strikethroughs (\sout)
                                % normalem makes italics be italics, not underlines
    \usepackage{mathrsfs}
    

    
    % Colors for the hyperref package
    \definecolor{urlcolor}{rgb}{0,.145,.698}
    \definecolor{linkcolor}{rgb}{.71,0.21,0.01}
    \definecolor{citecolor}{rgb}{.12,.54,.11}

    % ANSI colors
    \definecolor{ansi-black}{HTML}{3E424D}
    \definecolor{ansi-black-intense}{HTML}{282C36}
    \definecolor{ansi-red}{HTML}{E75C58}
    \definecolor{ansi-red-intense}{HTML}{B22B31}
    \definecolor{ansi-green}{HTML}{00A250}
    \definecolor{ansi-green-intense}{HTML}{007427}
    \definecolor{ansi-yellow}{HTML}{DDB62B}
    \definecolor{ansi-yellow-intense}{HTML}{B27D12}
    \definecolor{ansi-blue}{HTML}{208FFB}
    \definecolor{ansi-blue-intense}{HTML}{0065CA}
    \definecolor{ansi-magenta}{HTML}{D160C4}
    \definecolor{ansi-magenta-intense}{HTML}{A03196}
    \definecolor{ansi-cyan}{HTML}{60C6C8}
    \definecolor{ansi-cyan-intense}{HTML}{258F8F}
    \definecolor{ansi-white}{HTML}{C5C1B4}
    \definecolor{ansi-white-intense}{HTML}{A1A6B2}
    \definecolor{ansi-default-inverse-fg}{HTML}{FFFFFF}
    \definecolor{ansi-default-inverse-bg}{HTML}{000000}

    % commands and environments needed by pandoc snippets
    % extracted from the output of `pandoc -s`
    \providecommand{\tightlist}{%
      \setlength{\itemsep}{0pt}\setlength{\parskip}{0pt}}
    \DefineVerbatimEnvironment{Highlighting}{Verbatim}{commandchars=\\\{\}}
    % Add ',fontsize=\small' for more characters per line
    \newenvironment{Shaded}{}{}
    \newcommand{\KeywordTok}[1]{\textcolor[rgb]{0.00,0.44,0.13}{\textbf{{#1}}}}
    \newcommand{\DataTypeTok}[1]{\textcolor[rgb]{0.56,0.13,0.00}{{#1}}}
    \newcommand{\DecValTok}[1]{\textcolor[rgb]{0.25,0.63,0.44}{{#1}}}
    \newcommand{\BaseNTok}[1]{\textcolor[rgb]{0.25,0.63,0.44}{{#1}}}
    \newcommand{\FloatTok}[1]{\textcolor[rgb]{0.25,0.63,0.44}{{#1}}}
    \newcommand{\CharTok}[1]{\textcolor[rgb]{0.25,0.44,0.63}{{#1}}}
    \newcommand{\StringTok}[1]{\textcolor[rgb]{0.25,0.44,0.63}{{#1}}}
    \newcommand{\CommentTok}[1]{\textcolor[rgb]{0.38,0.63,0.69}{\textit{{#1}}}}
    \newcommand{\OtherTok}[1]{\textcolor[rgb]{0.00,0.44,0.13}{{#1}}}
    \newcommand{\AlertTok}[1]{\textcolor[rgb]{1.00,0.00,0.00}{\textbf{{#1}}}}
    \newcommand{\FunctionTok}[1]{\textcolor[rgb]{0.02,0.16,0.49}{{#1}}}
    \newcommand{\RegionMarkerTok}[1]{{#1}}
    \newcommand{\ErrorTok}[1]{\textcolor[rgb]{1.00,0.00,0.00}{\textbf{{#1}}}}
    \newcommand{\NormalTok}[1]{{#1}}
    
    % Additional commands for more recent versions of Pandoc
    \newcommand{\ConstantTok}[1]{\textcolor[rgb]{0.53,0.00,0.00}{{#1}}}
    \newcommand{\SpecialCharTok}[1]{\textcolor[rgb]{0.25,0.44,0.63}{{#1}}}
    \newcommand{\VerbatimStringTok}[1]{\textcolor[rgb]{0.25,0.44,0.63}{{#1}}}
    \newcommand{\SpecialStringTok}[1]{\textcolor[rgb]{0.73,0.40,0.53}{{#1}}}
    \newcommand{\ImportTok}[1]{{#1}}
    \newcommand{\DocumentationTok}[1]{\textcolor[rgb]{0.73,0.13,0.13}{\textit{{#1}}}}
    \newcommand{\AnnotationTok}[1]{\textcolor[rgb]{0.38,0.63,0.69}{\textbf{\textit{{#1}}}}}
    \newcommand{\CommentVarTok}[1]{\textcolor[rgb]{0.38,0.63,0.69}{\textbf{\textit{{#1}}}}}
    \newcommand{\VariableTok}[1]{\textcolor[rgb]{0.10,0.09,0.49}{{#1}}}
    \newcommand{\ControlFlowTok}[1]{\textcolor[rgb]{0.00,0.44,0.13}{\textbf{{#1}}}}
    \newcommand{\OperatorTok}[1]{\textcolor[rgb]{0.40,0.40,0.40}{{#1}}}
    \newcommand{\BuiltInTok}[1]{{#1}}
    \newcommand{\ExtensionTok}[1]{{#1}}
    \newcommand{\PreprocessorTok}[1]{\textcolor[rgb]{0.74,0.48,0.00}{{#1}}}
    \newcommand{\AttributeTok}[1]{\textcolor[rgb]{0.49,0.56,0.16}{{#1}}}
    \newcommand{\InformationTok}[1]{\textcolor[rgb]{0.38,0.63,0.69}{\textbf{\textit{{#1}}}}}
    \newcommand{\WarningTok}[1]{\textcolor[rgb]{0.38,0.63,0.69}{\textbf{\textit{{#1}}}}}
    
    
    % Define a nice break command that doesn't care if a line doesn't already
    % exist.
    \def\br{\hspace*{\fill} \\* }
    % Math Jax compatibility definitions
    \def\gt{>}
    \def\lt{<}
    \let\Oldtex\TeX
    \let\Oldlatex\LaTeX
    \renewcommand{\TeX}{\textrm{\Oldtex}}
    \renewcommand{\LaTeX}{\textrm{\Oldlatex}}
    % Document parameters
    % Document title
    \title{HW1}
    
    
    
    
    
% Pygments definitions
\makeatletter
\def\PY@reset{\let\PY@it=\relax \let\PY@bf=\relax%
    \let\PY@ul=\relax \let\PY@tc=\relax%
    \let\PY@bc=\relax \let\PY@ff=\relax}
\def\PY@tok#1{\csname PY@tok@#1\endcsname}
\def\PY@toks#1+{\ifx\relax#1\empty\else%
    \PY@tok{#1}\expandafter\PY@toks\fi}
\def\PY@do#1{\PY@bc{\PY@tc{\PY@ul{%
    \PY@it{\PY@bf{\PY@ff{#1}}}}}}}
\def\PY#1#2{\PY@reset\PY@toks#1+\relax+\PY@do{#2}}

\expandafter\def\csname PY@tok@w\endcsname{\def\PY@tc##1{\textcolor[rgb]{0.73,0.73,0.73}{##1}}}
\expandafter\def\csname PY@tok@c\endcsname{\let\PY@it=\textit\def\PY@tc##1{\textcolor[rgb]{0.25,0.50,0.50}{##1}}}
\expandafter\def\csname PY@tok@cp\endcsname{\def\PY@tc##1{\textcolor[rgb]{0.74,0.48,0.00}{##1}}}
\expandafter\def\csname PY@tok@k\endcsname{\let\PY@bf=\textbf\def\PY@tc##1{\textcolor[rgb]{0.00,0.50,0.00}{##1}}}
\expandafter\def\csname PY@tok@kp\endcsname{\def\PY@tc##1{\textcolor[rgb]{0.00,0.50,0.00}{##1}}}
\expandafter\def\csname PY@tok@kt\endcsname{\def\PY@tc##1{\textcolor[rgb]{0.69,0.00,0.25}{##1}}}
\expandafter\def\csname PY@tok@o\endcsname{\def\PY@tc##1{\textcolor[rgb]{0.40,0.40,0.40}{##1}}}
\expandafter\def\csname PY@tok@ow\endcsname{\let\PY@bf=\textbf\def\PY@tc##1{\textcolor[rgb]{0.67,0.13,1.00}{##1}}}
\expandafter\def\csname PY@tok@nb\endcsname{\def\PY@tc##1{\textcolor[rgb]{0.00,0.50,0.00}{##1}}}
\expandafter\def\csname PY@tok@nf\endcsname{\def\PY@tc##1{\textcolor[rgb]{0.00,0.00,1.00}{##1}}}
\expandafter\def\csname PY@tok@nc\endcsname{\let\PY@bf=\textbf\def\PY@tc##1{\textcolor[rgb]{0.00,0.00,1.00}{##1}}}
\expandafter\def\csname PY@tok@nn\endcsname{\let\PY@bf=\textbf\def\PY@tc##1{\textcolor[rgb]{0.00,0.00,1.00}{##1}}}
\expandafter\def\csname PY@tok@ne\endcsname{\let\PY@bf=\textbf\def\PY@tc##1{\textcolor[rgb]{0.82,0.25,0.23}{##1}}}
\expandafter\def\csname PY@tok@nv\endcsname{\def\PY@tc##1{\textcolor[rgb]{0.10,0.09,0.49}{##1}}}
\expandafter\def\csname PY@tok@no\endcsname{\def\PY@tc##1{\textcolor[rgb]{0.53,0.00,0.00}{##1}}}
\expandafter\def\csname PY@tok@nl\endcsname{\def\PY@tc##1{\textcolor[rgb]{0.63,0.63,0.00}{##1}}}
\expandafter\def\csname PY@tok@ni\endcsname{\let\PY@bf=\textbf\def\PY@tc##1{\textcolor[rgb]{0.60,0.60,0.60}{##1}}}
\expandafter\def\csname PY@tok@na\endcsname{\def\PY@tc##1{\textcolor[rgb]{0.49,0.56,0.16}{##1}}}
\expandafter\def\csname PY@tok@nt\endcsname{\let\PY@bf=\textbf\def\PY@tc##1{\textcolor[rgb]{0.00,0.50,0.00}{##1}}}
\expandafter\def\csname PY@tok@nd\endcsname{\def\PY@tc##1{\textcolor[rgb]{0.67,0.13,1.00}{##1}}}
\expandafter\def\csname PY@tok@s\endcsname{\def\PY@tc##1{\textcolor[rgb]{0.73,0.13,0.13}{##1}}}
\expandafter\def\csname PY@tok@sd\endcsname{\let\PY@it=\textit\def\PY@tc##1{\textcolor[rgb]{0.73,0.13,0.13}{##1}}}
\expandafter\def\csname PY@tok@si\endcsname{\let\PY@bf=\textbf\def\PY@tc##1{\textcolor[rgb]{0.73,0.40,0.53}{##1}}}
\expandafter\def\csname PY@tok@se\endcsname{\let\PY@bf=\textbf\def\PY@tc##1{\textcolor[rgb]{0.73,0.40,0.13}{##1}}}
\expandafter\def\csname PY@tok@sr\endcsname{\def\PY@tc##1{\textcolor[rgb]{0.73,0.40,0.53}{##1}}}
\expandafter\def\csname PY@tok@ss\endcsname{\def\PY@tc##1{\textcolor[rgb]{0.10,0.09,0.49}{##1}}}
\expandafter\def\csname PY@tok@sx\endcsname{\def\PY@tc##1{\textcolor[rgb]{0.00,0.50,0.00}{##1}}}
\expandafter\def\csname PY@tok@m\endcsname{\def\PY@tc##1{\textcolor[rgb]{0.40,0.40,0.40}{##1}}}
\expandafter\def\csname PY@tok@gh\endcsname{\let\PY@bf=\textbf\def\PY@tc##1{\textcolor[rgb]{0.00,0.00,0.50}{##1}}}
\expandafter\def\csname PY@tok@gu\endcsname{\let\PY@bf=\textbf\def\PY@tc##1{\textcolor[rgb]{0.50,0.00,0.50}{##1}}}
\expandafter\def\csname PY@tok@gd\endcsname{\def\PY@tc##1{\textcolor[rgb]{0.63,0.00,0.00}{##1}}}
\expandafter\def\csname PY@tok@gi\endcsname{\def\PY@tc##1{\textcolor[rgb]{0.00,0.63,0.00}{##1}}}
\expandafter\def\csname PY@tok@gr\endcsname{\def\PY@tc##1{\textcolor[rgb]{1.00,0.00,0.00}{##1}}}
\expandafter\def\csname PY@tok@ge\endcsname{\let\PY@it=\textit}
\expandafter\def\csname PY@tok@gs\endcsname{\let\PY@bf=\textbf}
\expandafter\def\csname PY@tok@gp\endcsname{\let\PY@bf=\textbf\def\PY@tc##1{\textcolor[rgb]{0.00,0.00,0.50}{##1}}}
\expandafter\def\csname PY@tok@go\endcsname{\def\PY@tc##1{\textcolor[rgb]{0.53,0.53,0.53}{##1}}}
\expandafter\def\csname PY@tok@gt\endcsname{\def\PY@tc##1{\textcolor[rgb]{0.00,0.27,0.87}{##1}}}
\expandafter\def\csname PY@tok@err\endcsname{\def\PY@bc##1{\setlength{\fboxsep}{0pt}\fcolorbox[rgb]{1.00,0.00,0.00}{1,1,1}{\strut ##1}}}
\expandafter\def\csname PY@tok@kc\endcsname{\let\PY@bf=\textbf\def\PY@tc##1{\textcolor[rgb]{0.00,0.50,0.00}{##1}}}
\expandafter\def\csname PY@tok@kd\endcsname{\let\PY@bf=\textbf\def\PY@tc##1{\textcolor[rgb]{0.00,0.50,0.00}{##1}}}
\expandafter\def\csname PY@tok@kn\endcsname{\let\PY@bf=\textbf\def\PY@tc##1{\textcolor[rgb]{0.00,0.50,0.00}{##1}}}
\expandafter\def\csname PY@tok@kr\endcsname{\let\PY@bf=\textbf\def\PY@tc##1{\textcolor[rgb]{0.00,0.50,0.00}{##1}}}
\expandafter\def\csname PY@tok@bp\endcsname{\def\PY@tc##1{\textcolor[rgb]{0.00,0.50,0.00}{##1}}}
\expandafter\def\csname PY@tok@fm\endcsname{\def\PY@tc##1{\textcolor[rgb]{0.00,0.00,1.00}{##1}}}
\expandafter\def\csname PY@tok@vc\endcsname{\def\PY@tc##1{\textcolor[rgb]{0.10,0.09,0.49}{##1}}}
\expandafter\def\csname PY@tok@vg\endcsname{\def\PY@tc##1{\textcolor[rgb]{0.10,0.09,0.49}{##1}}}
\expandafter\def\csname PY@tok@vi\endcsname{\def\PY@tc##1{\textcolor[rgb]{0.10,0.09,0.49}{##1}}}
\expandafter\def\csname PY@tok@vm\endcsname{\def\PY@tc##1{\textcolor[rgb]{0.10,0.09,0.49}{##1}}}
\expandafter\def\csname PY@tok@sa\endcsname{\def\PY@tc##1{\textcolor[rgb]{0.73,0.13,0.13}{##1}}}
\expandafter\def\csname PY@tok@sb\endcsname{\def\PY@tc##1{\textcolor[rgb]{0.73,0.13,0.13}{##1}}}
\expandafter\def\csname PY@tok@sc\endcsname{\def\PY@tc##1{\textcolor[rgb]{0.73,0.13,0.13}{##1}}}
\expandafter\def\csname PY@tok@dl\endcsname{\def\PY@tc##1{\textcolor[rgb]{0.73,0.13,0.13}{##1}}}
\expandafter\def\csname PY@tok@s2\endcsname{\def\PY@tc##1{\textcolor[rgb]{0.73,0.13,0.13}{##1}}}
\expandafter\def\csname PY@tok@sh\endcsname{\def\PY@tc##1{\textcolor[rgb]{0.73,0.13,0.13}{##1}}}
\expandafter\def\csname PY@tok@s1\endcsname{\def\PY@tc##1{\textcolor[rgb]{0.73,0.13,0.13}{##1}}}
\expandafter\def\csname PY@tok@mb\endcsname{\def\PY@tc##1{\textcolor[rgb]{0.40,0.40,0.40}{##1}}}
\expandafter\def\csname PY@tok@mf\endcsname{\def\PY@tc##1{\textcolor[rgb]{0.40,0.40,0.40}{##1}}}
\expandafter\def\csname PY@tok@mh\endcsname{\def\PY@tc##1{\textcolor[rgb]{0.40,0.40,0.40}{##1}}}
\expandafter\def\csname PY@tok@mi\endcsname{\def\PY@tc##1{\textcolor[rgb]{0.40,0.40,0.40}{##1}}}
\expandafter\def\csname PY@tok@il\endcsname{\def\PY@tc##1{\textcolor[rgb]{0.40,0.40,0.40}{##1}}}
\expandafter\def\csname PY@tok@mo\endcsname{\def\PY@tc##1{\textcolor[rgb]{0.40,0.40,0.40}{##1}}}
\expandafter\def\csname PY@tok@ch\endcsname{\let\PY@it=\textit\def\PY@tc##1{\textcolor[rgb]{0.25,0.50,0.50}{##1}}}
\expandafter\def\csname PY@tok@cm\endcsname{\let\PY@it=\textit\def\PY@tc##1{\textcolor[rgb]{0.25,0.50,0.50}{##1}}}
\expandafter\def\csname PY@tok@cpf\endcsname{\let\PY@it=\textit\def\PY@tc##1{\textcolor[rgb]{0.25,0.50,0.50}{##1}}}
\expandafter\def\csname PY@tok@c1\endcsname{\let\PY@it=\textit\def\PY@tc##1{\textcolor[rgb]{0.25,0.50,0.50}{##1}}}
\expandafter\def\csname PY@tok@cs\endcsname{\let\PY@it=\textit\def\PY@tc##1{\textcolor[rgb]{0.25,0.50,0.50}{##1}}}

\def\PYZbs{\char`\\}
\def\PYZus{\char`\_}
\def\PYZob{\char`\{}
\def\PYZcb{\char`\}}
\def\PYZca{\char`\^}
\def\PYZam{\char`\&}
\def\PYZlt{\char`\<}
\def\PYZgt{\char`\>}
\def\PYZsh{\char`\#}
\def\PYZpc{\char`\%}
\def\PYZdl{\char`\$}
\def\PYZhy{\char`\-}
\def\PYZsq{\char`\'}
\def\PYZdq{\char`\"}
\def\PYZti{\char`\~}
% for compatibility with earlier versions
\def\PYZat{@}
\def\PYZlb{[}
\def\PYZrb{]}
\makeatother


    % For linebreaks inside Verbatim environment from package fancyvrb. 
    \makeatletter
        \newbox\Wrappedcontinuationbox 
        \newbox\Wrappedvisiblespacebox 
        \newcommand*\Wrappedvisiblespace {\textcolor{red}{\textvisiblespace}} 
        \newcommand*\Wrappedcontinuationsymbol {\textcolor{red}{\llap{\tiny$\m@th\hookrightarrow$}}} 
        \newcommand*\Wrappedcontinuationindent {3ex } 
        \newcommand*\Wrappedafterbreak {\kern\Wrappedcontinuationindent\copy\Wrappedcontinuationbox} 
        % Take advantage of the already applied Pygments mark-up to insert 
        % potential linebreaks for TeX processing. 
        %        {, <, #, %, $, ' and ": go to next line. 
        %        _, }, ^, &, >, - and ~: stay at end of broken line. 
        % Use of \textquotesingle for straight quote. 
        \newcommand*\Wrappedbreaksatspecials {% 
            \def\PYGZus{\discretionary{\char`\_}{\Wrappedafterbreak}{\char`\_}}% 
            \def\PYGZob{\discretionary{}{\Wrappedafterbreak\char`\{}{\char`\{}}% 
            \def\PYGZcb{\discretionary{\char`\}}{\Wrappedafterbreak}{\char`\}}}% 
            \def\PYGZca{\discretionary{\char`\^}{\Wrappedafterbreak}{\char`\^}}% 
            \def\PYGZam{\discretionary{\char`\&}{\Wrappedafterbreak}{\char`\&}}% 
            \def\PYGZlt{\discretionary{}{\Wrappedafterbreak\char`\<}{\char`\<}}% 
            \def\PYGZgt{\discretionary{\char`\>}{\Wrappedafterbreak}{\char`\>}}% 
            \def\PYGZsh{\discretionary{}{\Wrappedafterbreak\char`\#}{\char`\#}}% 
            \def\PYGZpc{\discretionary{}{\Wrappedafterbreak\char`\%}{\char`\%}}% 
            \def\PYGZdl{\discretionary{}{\Wrappedafterbreak\char`\$}{\char`\$}}% 
            \def\PYGZhy{\discretionary{\char`\-}{\Wrappedafterbreak}{\char`\-}}% 
            \def\PYGZsq{\discretionary{}{\Wrappedafterbreak\textquotesingle}{\textquotesingle}}% 
            \def\PYGZdq{\discretionary{}{\Wrappedafterbreak\char`\"}{\char`\"}}% 
            \def\PYGZti{\discretionary{\char`\~}{\Wrappedafterbreak}{\char`\~}}% 
        } 
        % Some characters . , ; ? ! / are not pygmentized. 
        % This macro makes them "active" and they will insert potential linebreaks 
        \newcommand*\Wrappedbreaksatpunct {% 
            \lccode`\~`\.\lowercase{\def~}{\discretionary{\hbox{\char`\.}}{\Wrappedafterbreak}{\hbox{\char`\.}}}% 
            \lccode`\~`\,\lowercase{\def~}{\discretionary{\hbox{\char`\,}}{\Wrappedafterbreak}{\hbox{\char`\,}}}% 
            \lccode`\~`\;\lowercase{\def~}{\discretionary{\hbox{\char`\;}}{\Wrappedafterbreak}{\hbox{\char`\;}}}% 
            \lccode`\~`\:\lowercase{\def~}{\discretionary{\hbox{\char`\:}}{\Wrappedafterbreak}{\hbox{\char`\:}}}% 
            \lccode`\~`\?\lowercase{\def~}{\discretionary{\hbox{\char`\?}}{\Wrappedafterbreak}{\hbox{\char`\?}}}% 
            \lccode`\~`\!\lowercase{\def~}{\discretionary{\hbox{\char`\!}}{\Wrappedafterbreak}{\hbox{\char`\!}}}% 
            \lccode`\~`\/\lowercase{\def~}{\discretionary{\hbox{\char`\/}}{\Wrappedafterbreak}{\hbox{\char`\/}}}% 
            \catcode`\.\active
            \catcode`\,\active 
            \catcode`\;\active
            \catcode`\:\active
            \catcode`\?\active
            \catcode`\!\active
            \catcode`\/\active 
            \lccode`\~`\~ 	
        }
    \makeatother

    \let\OriginalVerbatim=\Verbatim
    \makeatletter
    \renewcommand{\Verbatim}[1][1]{%
        %\parskip\z@skip
        \sbox\Wrappedcontinuationbox {\Wrappedcontinuationsymbol}%
        \sbox\Wrappedvisiblespacebox {\FV@SetupFont\Wrappedvisiblespace}%
        \def\FancyVerbFormatLine ##1{\hsize\linewidth
            \vtop{\raggedright\hyphenpenalty\z@\exhyphenpenalty\z@
                \doublehyphendemerits\z@\finalhyphendemerits\z@
                \strut ##1\strut}%
        }%
        % If the linebreak is at a space, the latter will be displayed as visible
        % space at end of first line, and a continuation symbol starts next line.
        % Stretch/shrink are however usually zero for typewriter font.
        \def\FV@Space {%
            \nobreak\hskip\z@ plus\fontdimen3\font minus\fontdimen4\font
            \discretionary{\copy\Wrappedvisiblespacebox}{\Wrappedafterbreak}
            {\kern\fontdimen2\font}%
        }%
        
        % Allow breaks at special characters using \PYG... macros.
        \Wrappedbreaksatspecials
        % Breaks at punctuation characters . , ; ? ! and / need catcode=\active 	
        \OriginalVerbatim[#1,codes*=\Wrappedbreaksatpunct]%
    }
    \makeatother

    % Exact colors from NB
    \definecolor{incolor}{HTML}{303F9F}
    \definecolor{outcolor}{HTML}{D84315}
    \definecolor{cellborder}{HTML}{CFCFCF}
    \definecolor{cellbackground}{HTML}{F7F7F7}
    
    % prompt
    \makeatletter
    \newcommand{\boxspacing}{\kern\kvtcb@left@rule\kern\kvtcb@boxsep}
    \makeatother
    \newcommand{\prompt}[4]{
        \ttfamily\llap{{\color{#2}[#3]:\hspace{3pt}#4}}\vspace{-\baselineskip}
    }
    

    
    % Prevent overflowing lines due to hard-to-break entities
    \sloppy 
    % Setup hyperref package
    \hypersetup{
      breaklinks=true,  % so long urls are correctly broken across lines
      colorlinks=true,
      urlcolor=urlcolor,
      linkcolor=linkcolor,
      citecolor=citecolor,
      }
    % Slightly bigger margins than the latex defaults
    
    \geometry{verbose,tmargin=1in,bmargin=1in,lmargin=1in,rmargin=1in}
    
    

\begin{document}
    
    \maketitle
    
    

    
    \begin{tcolorbox}[breakable, size=fbox, boxrule=1pt, pad at break*=1mm,colback=cellbackground, colframe=cellborder]
\prompt{In}{incolor}{13}{\boxspacing}
\begin{Verbatim}[commandchars=\\\{\}]
\PY{n+nf}{library}\PY{p}{(}\PY{n}{ggplot2}\PY{p}{)}
\PY{n+nf}{set.seed}\PY{p}{(}\PY{l+m}{1}\PY{p}{)}
\end{Verbatim}
\end{tcolorbox}

    \hypertarget{section}{%
\section{1.5}\label{section}}

    \begin{tcolorbox}[breakable, size=fbox, boxrule=1pt, pad at break*=1mm,colback=cellbackground, colframe=cellborder]
\prompt{In}{incolor}{25}{\boxspacing}
\begin{Verbatim}[commandchars=\\\{\}]
\PY{n}{m} \PY{o}{\PYZlt{}\PYZhy{}} \PY{l+m}{10000}
\end{Verbatim}
\end{tcolorbox}

    \begin{tcolorbox}[breakable, size=fbox, boxrule=1pt, pad at break*=1mm,colback=cellbackground, colframe=cellborder]
\prompt{In}{incolor}{42}{\boxspacing}
\begin{Verbatim}[commandchars=\\\{\}]
\PY{c+c1}{\PYZsh{} approach a.}
\PY{n}{x} \PY{o}{\PYZlt{}\PYZhy{}} \PY{n+nf}{rcauchy}\PY{p}{(}\PY{n}{m}\PY{p}{,} \PY{l+m}{0}\PY{p}{,} \PY{l+m}{1}\PY{p}{)}
\PY{n}{theta} \PY{o}{\PYZlt{}\PYZhy{}} \PY{n+nf}{sum}\PY{p}{(}\PY{n}{x} \PY{o}{\PYZgt{}} \PY{l+m}{2}\PY{p}{)} \PY{o}{/} \PY{n}{m}
\PY{n}{theta}
\end{Verbatim}
\end{tcolorbox}

    0.1449

    
    \(X\sim \frac{1}{\pi(1+x^2)}\). The expectation of \(\phi(X)\) is
\[\begin{align}
    E(\phi(X)) &= 1\times P(\phi(X)=1)+0\times P(\phi(X)=0) \\
    &= P(\phi(X)=1) \\
    &= P(X>2) \\
    &= \theta
\end{align}\]

    \begin{tcolorbox}[breakable, size=fbox, boxrule=1pt, pad at break*=1mm,colback=cellbackground, colframe=cellborder]
\prompt{In}{incolor}{46}{\boxspacing}
\begin{Verbatim}[commandchars=\\\{\}]
\PY{c+c1}{\PYZsh{} standard deviation}
\PY{n}{sd} \PY{o}{\PYZlt{}\PYZhy{}} \PY{n+nf}{sqrt}\PY{p}{(}\PY{l+m}{1}\PY{o}{/}\PY{p}{(}\PY{n}{m}\PY{l+m}{\PYZhy{}1}\PY{p}{)} \PY{o}{*} \PY{n+nf}{sum}\PY{p}{(}\PY{p}{(}\PY{p}{(}\PY{n}{x}\PY{o}{\PYZgt{}}\PY{l+m}{2}\PY{p}{)}\PY{o}{\PYZhy{}}\PY{n}{theta}\PY{p}{)}\PY{o}{\PYZca{}}\PY{l+m}{2}\PY{p}{)}\PY{p}{)}
\PY{n}{sd}
\end{Verbatim}
\end{tcolorbox}

    0.352029533532697

    
    \begin{tcolorbox}[breakable, size=fbox, boxrule=1pt, pad at break*=1mm,colback=cellbackground, colframe=cellborder]
\prompt{In}{incolor}{44}{\boxspacing}
\begin{Verbatim}[commandchars=\\\{\}]
\PY{c+c1}{\PYZsh{} approach b.}
\PY{n}{theta} \PY{o}{\PYZlt{}\PYZhy{}} \PY{l+m}{1} \PY{o}{/} \PY{l+m}{2} \PY{o}{*} \PY{n+nf}{sum}\PY{p}{(}\PY{n+nf}{abs}\PY{p}{(}\PY{n}{x}\PY{p}{)} \PY{o}{\PYZgt{}} \PY{l+m}{2}\PY{p}{)} \PY{o}{/} \PY{n}{m}
\PY{n}{theta}
\end{Verbatim}
\end{tcolorbox}

    0.1478

    
    \(X\sim \frac{1}{\pi(1+x^2)}\). The expectation of \(\phi(X)\) is
\[\begin{align}
    E(\phi(X)) &= 1/2\times P(\phi(X)=1/2)+0\times P(\phi(X)=0) \\
    &= 1/2\times (P(X>2)+P(X<-2)) \\
    &= P(X>2) \\
    &= \theta
\end{align}\]

    \begin{tcolorbox}[breakable, size=fbox, boxrule=1pt, pad at break*=1mm,colback=cellbackground, colframe=cellborder]
\prompt{In}{incolor}{51}{\boxspacing}
\begin{Verbatim}[commandchars=\\\{\}]
\PY{c+c1}{\PYZsh{} standard deviation}
\PY{n}{sd} \PY{o}{\PYZlt{}\PYZhy{}} \PY{n+nf}{sqrt}\PY{p}{(}\PY{l+m}{1}\PY{o}{/}\PY{p}{(}\PY{n}{m}\PY{l+m}{\PYZhy{}1}\PY{p}{)} \PY{o}{*} \PY{n+nf}{sum}\PY{p}{(}\PY{p}{(}\PY{l+m}{1}\PY{o}{/}\PY{l+m}{2}\PY{o}{*}\PY{p}{(}\PY{n+nf}{abs}\PY{p}{(}\PY{n}{x}\PY{p}{)}\PY{o}{\PYZgt{}}\PY{l+m}{2}\PY{p}{)}\PY{o}{\PYZhy{}}\PY{n}{theta}\PY{p}{)}\PY{o}{\PYZca{}}\PY{l+m}{2}\PY{p}{)}\PY{p}{)}
\PY{n}{sd}
\end{Verbatim}
\end{tcolorbox}

    0.147642230833806

    
    \begin{tcolorbox}[breakable, size=fbox, boxrule=1pt, pad at break*=1mm,colback=cellbackground, colframe=cellborder]
\prompt{In}{incolor}{50}{\boxspacing}
\begin{Verbatim}[commandchars=\\\{\}]
\PY{c+c1}{\PYZsh{} approach c.}
\PY{n}{x} \PY{o}{\PYZlt{}\PYZhy{}} \PY{n+nf}{runif}\PY{p}{(}\PY{n}{m}\PY{p}{,} \PY{l+m}{0}\PY{p}{,} \PY{l+m}{1}\PY{o}{/}\PY{l+m}{2}\PY{p}{)}
\PY{n}{theta} \PY{o}{\PYZlt{}\PYZhy{}} \PY{l+m}{1} \PY{o}{/} \PY{l+m}{2} \PY{o}{*} \PY{n+nf}{sum}\PY{p}{(}\PY{n+nf}{dcauchy}\PY{p}{(}\PY{n}{x}\PY{p}{,} \PY{l+m}{0}\PY{p}{,} \PY{l+m}{1}\PY{p}{)}\PY{p}{)} \PY{o}{/} \PY{n}{m}
\PY{n}{theta}
\end{Verbatim}
\end{tcolorbox}

    0.147634848537703

    
    \(X\sim U(0,1/2)\). The expectation of \(\phi(X)\) is \[\begin{align}
    E(\phi(X)) &= 1/2\times E(f(X)) \\
    &= 1/2\times  \int_0^{1/2} \frac{2}{\pi(1+x^2)}dx \\
\end{align}\] Let \(t=1/x\), the above integration becomes
\[\begin{align}
    &= \int_{\infty}^{2} \frac{-1}{\pi(1+t^2)}dt   \\
    &= \int_{2}^{\infty} \frac{1}{\pi(1+t^2)}dt \\
    &= P(Y>2) \quad where\ Y\sim \frac{1}{\pi(1+y^2)}\\
    &= \theta
\end{align}\]

    \begin{tcolorbox}[breakable, size=fbox, boxrule=1pt, pad at break*=1mm,colback=cellbackground, colframe=cellborder]
\prompt{In}{incolor}{55}{\boxspacing}
\begin{Verbatim}[commandchars=\\\{\}]
\PY{c+c1}{\PYZsh{} standard deviation}
\PY{n}{sd} \PY{o}{\PYZlt{}\PYZhy{}} \PY{n+nf}{sqrt}\PY{p}{(}\PY{l+m}{1}\PY{o}{/}\PY{p}{(}\PY{n}{m}\PY{l+m}{\PYZhy{}1}\PY{p}{)} \PY{o}{*} \PY{n+nf}{sum}\PY{p}{(}\PY{p}{(}\PY{l+m}{1}\PY{o}{/}\PY{l+m}{2}\PY{o}{*}\PY{n+nf}{dcauchy}\PY{p}{(}\PY{n}{x}\PY{p}{,} \PY{l+m}{0}\PY{p}{,} \PY{l+m}{1}\PY{p}{)}\PY{o}{\PYZhy{}}\PY{n}{theta}\PY{p}{)}\PY{o}{\PYZca{}}\PY{l+m}{2}\PY{p}{)}\PY{p}{)}
\PY{n}{sd}
\end{Verbatim}
\end{tcolorbox}

    0.00983345846102847

    
    Approach c has the smallest standard deviation while approach a has the
largest one.

    \hypertarget{section}{%
\section{2.4}\label{section}}

    The noninformative prior is
\(\pi(\mu,\sigma_x^2)\propto \frac{1}{\sigma_x^2}\),
\(\pi(\lambda,\sigma_y^2)\propto \frac{1}{\sigma_y^2}\). We integrate
\(\mu\) and \(\lambda\) to obtain the posterior of \(\sigma_x^2\) and
\(\sigma_y^2\) \[\begin{align}
    p(\sigma_x^2,\sigma_y^2|X,Y)&\propto \int\int \sigma_x^{-n-2}\exp(\frac{-\sum(x_i-\mu)^2}{2\sigma_x^2})\sigma_y^{-m-2}\exp(\frac{-\sum(y_i-\lambda)^2}{2\sigma_y^2}) d\mu d\lambda\\
    &= \int\int \sigma_x^{-n-2}\exp(\frac{-\sum(x_i-\bar{x})^2-n(\bar{x}-\mu)^2}{2\sigma_x^2})\sigma_y^{-m-2}\exp(\frac{-\sum(y_i-\bar{y})^2-m(\bar{y}-\lambda)^2}{2\sigma_y^2}) d\mu d\lambda \\
    &= \sigma_x^{-n-1}\exp(\frac{-\sum(x_i-\bar{x})^2}{2\sigma_x^2})\sigma_y^{-m-1}\exp(\frac{-\sum(y_i-\bar{y})^2}{2\sigma_y^2}) \\
    &= \sigma_x^{-n-1}\exp(\frac{-(n-1)s_x^2}{2\sigma_x^2})\sigma_y^{-m-1}\exp(\frac{-(m-1)s_y^2}{2\sigma_y^2})
\end{align}\] Since \(\sigma_x^2\) and \(\sigma_y^2\) are independent,
they respectively have the inverse gamma posterior distribution as
\[\begin{align}
    p(\sigma_x^2|X,Y)&\propto InverseGamma(\frac{n-1}{2}, \frac{(n-1)s_x^2}{2}) \\
    p(\sigma_y^2|X,Y)&\propto InverseGamma(\frac{m-1}{2}, \frac{(m-1)s_y^2}{2})
\end{align}\] Equivalently, \[\begin{align}
    p(\frac{1}{\sigma_x^2}|X,Y)&\propto \Gamma(\frac{n-1}{2}, \frac{(n-1)s_x^2}{2}) \\
    p(\frac{1}{\sigma_y^2}|X,Y)&\propto \Gamma(\frac{m-1}{2}, \frac{(m-1)s_y^2}{2})
\end{align}\] Therefore,
\(\frac{(n-1)s_x^2}{\sigma_x^2}\sim \chi^2(n-1)\),
\(\frac{(m-1)s_y^2}{\sigma_y^2}\sim \chi^2(m-1)\) and they are
independent. So \[\begin{align}
    F=\frac{\frac{(n-1)s_x^2}{\sigma_x^2(n-1)}}{\frac{(m-1)s_y^2}{\sigma_y^2(m-1)}}=\frac{\sigma_y^2/s_y^2}{\sigma_x^2/s_x^2}\sim F(n-1,m-1)
\end{align}\]

    \hypertarget{section}{%
\section{2.6}\label{section}}

    \(\theta=(\mu, \sigma^2)\), the likelihood is \[\begin{align}
    L(Y|\theta) &= \frac{1}{(2\pi\sigma^2)^{n/2}}\exp(\frac{-\sum(y_i-\mu)^2}{2\sigma^2}) \\
    l(Y|\theta) &= \frac{-n}{2}\log(2\pi\sigma^2) - \frac{\sum(y_i-\mu)^2}{2\sigma^2} \\
\end{align}\] The first partial derivatives is \[\begin{align}
    \frac{\partial l}{\partial \mu}&= \frac{\sum(y_i-\mu)}{\sigma^2}\\
    \frac{\partial l}{\partial \sigma^2}&= \frac{-n}{2\sigma^2}+\frac{\sum(y_i-\mu)^2}{2\sigma^4}
\end{align}\] Since Fisher score has the property \[\begin{align}
    E\left[\left(\frac{\partial l}{\partial \theta}\right)^2\right]= -E\left[\frac{\partial^2 l}{\partial \theta^2}\right]
\end{align}\] We can calculate \[\begin{align}
    \begin{pmatrix} \frac{\sum(y_i-\mu)}{\sigma^2} \\ \frac{-n}{2\sigma^2}+\frac{\sum(y_i-\mu)^2}{2\sigma^4} \end{pmatrix} \begin{pmatrix} \frac{\sum(y_i-\mu)}{\sigma^2} & \frac{-n}{2\sigma^2}+\frac{\sum(y_i-\mu)^2}{2\sigma^4} \end{pmatrix} &= \begin{pmatrix} \frac{[\sum(y_i-\mu)]^2}{\sigma^4} & \frac{-n\sum(y_i-\mu)}{2\sigma^4}+\frac{\sum(y_i-\mu)\sum(y_i-\mu)^2}{2\sigma^6}\\ \frac{-n\sum(y_i-\mu)}{2\sigma^4}+\frac{\sum(y_i-\mu)\sum(y_i-\mu)^2}{2\sigma^6} & \frac{n^2}{4\sigma^4}-\frac{n\sum(y_i-\mu)^2}{2\sigma^6} + \frac{[\sum(y_i-\mu)^2]^2}{4\sigma^8}\end{pmatrix}
\end{align}\] Then we calculate the expectation respectively
\[\begin{align}
    E\left(\frac{[\sum(y_i-\mu)]^2}{\sigma^4}\right)&= \frac{E\left((\sum y_i)^2-2n\mu\sum y_i+n^2\mu^2\right)}{\sigma^4} \\
    &= \frac{E((\sum y_i)^2)-n^2\mu^2}{\sigma^4} \\
    &= \frac{Var(\sum y_i)+(E(\sum y_i))^2-n^2\mu^2}{\sigma^4} \\
    &= \frac{n\sigma^2+n^2\mu^2-n^2\mu^2}{\sigma^4} \\
    &= \frac{n}{\sigma^2}
\end{align}\] and \[\begin{align}
    &E\left(\frac{-n\sum(y_i-\mu)}{2\sigma^4} + \frac{\sum(y_i-\mu)\sum(y_i-\mu)^2}{2\sigma^6}\right) = E\left( \frac{\sum(y_i-\mu)\sum(y_i-\mu)^2}{2\sigma^6}\right) \\
    &= \frac{E\left((\sum y_i-n\mu)(\sum y_i^2-2\mu\sum y_i+n\mu^2)\right)}{2\sigma^6} \\
    &= \frac{E(\sum y_i\sum y_i^2) -n\mu E(\sum y_i^2) -2\mu E((\sum y_i)^2)+2n\mu^2E(\sum y_i)+n\mu^2E(\sum y_i)-n^2\mu^3}{2\sigma^6} \\
    &= \frac{E(\sum y_i^2+\sum_{i\neq j}y_iy_j^2) - n\mu(n\mu^2+n\sigma^2)-2\mu(n^2\mu^2+n^2\sigma^2)+2n^2\mu^3+n^2\mu^3-n^2\mu^3}{2\sigma^6} \\
    &= \frac{n\mu^3+3n\mu\sigma^2+n(n-1)\mu(\mu^2+\sigma^2)- n\mu(n\mu^2+n\sigma^2)-2\mu n\sigma^2}{2\sigma^6} \\
    &= 0
\end{align}\] and \[\begin{align}
    E(\frac{n^2}{4\sigma^4}-\frac{n\sum(y_i-\mu)^2}{2\sigma^6} + \frac{[\sum(y_i-\mu)^2]^2}{4\sigma^8}) &= \frac{n^2}{4\sigma^4} - \frac{n^2E(((y-\mu))^2)}{2\sigma^6} + \frac{E((\sum(y_i-\mu)^2)^2)}{4\sigma^8} \\
    &= \frac{n^2}{4\sigma^4} - \frac{n^2\sigma^2}{2\sigma^6} + \frac{(E(\sum (y_i-\mu)^2))^2+Var(\sum (y_i-mu)^2)}{4\sigma^8} \\
    &= \frac{n^2}{4\sigma^4} - \frac{n^2\sigma^2}{2\sigma^6} + \frac{\left[nE((y-\mu)^2)\right]^2+nVar((y_i-\mu)^2)}{4\sigma^8} \\
    &= \frac{n^2}{4\sigma^4} - \frac{n^2\sigma^2}{2\sigma^6} + \frac{n^2\sigma^4+n\left(E((y-\mu)^4)-[E((y-\mu)^2)]^2\right)}{4\sigma^8} \\
    &= \frac{n^2}{4\sigma^4} - \frac{n^2\sigma^2}{2\sigma^6} + \frac{n^2\sigma^4+n(3\sigma^4-\sigma^4)}{4\sigma^8} \\
    &= \frac{n}{2\sigma^4}
\end{align}\] The fisher information matrix is
\[\left( \begin{array}{ccc}
    \frac{n}{\sigma^2} & 0 \\
    0 &  \frac{n}{2\sigma^4}
\end{array}\right)\] so the invariant prior is \[\begin{align}
    \pi(\mu, \sigma^2)\propto \frac{1}{\sigma^3}
\end{align}\]

    \hypertarget{section}{%
\section{2.12}\label{section}}

    The likelihood is \[\begin{align}
    L(Y|\theta,\sigma^2)&=(\frac{1}{\sqrt{2\pi}})^n\sigma^{-n}\exp\{\frac{-1}{2\sigma^2}(Y-X\theta)'(Y-X\theta)\} \\
    &= (\frac{1}{\sqrt{2\pi}})^n\sigma^{-n}\exp\{\frac{-1}{2\sigma^2}(vs^2+(\theta-\hat{\theta})'X'X(\theta-\hat{\theta}))\}
\end{align}\] where \(v=n-d\), \(s^2=(Y-\hat{Y})'(Y-\hat{Y})/v\),
\(vs^2=\)SSE.\\
Since we need to obtain the marginal posterior distribution of
\(\theta\), we need to integrate out \(\sigma^2\) as \[\begin{align}
    p(\theta|Y)=\int p(\theta,\sigma^2|Y)d\sigma^2
\end{align}\] The joint posterior is \[\begin{align}
    P(\theta,\sigma^2|Y)&\propto L(Y|\theta,\sigma^2)\pi(\theta,\sigma^2) \\
    &\propto \sigma^{-n-2}\exp\{\frac{-1}{2\sigma^2}(vs^2+(\theta-\hat{\theta})'X'X(\theta-\hat{\theta}))\}
\end{align}\] The marginal posterior distribution of \(\theta\) is
\[\begin{align}
    p(\theta|Y)&\propto \int_0^{\infty}  \sigma^{-n-2}\exp\{\frac{-1}{2\sigma^2}(vs^2+(\theta-\hat{\theta})'X'X(\theta-\hat{\theta}))\} d\sigma^2 \\
    &= \int_0^\infty \phi^{\frac{n-2}{2}}\exp\{\frac{-1}{2}(vs^2+(\theta-\hat{\theta})'X'X(\theta-\hat{\theta}))\phi\} d\phi \quad (\phi=\frac{1}{\sigma^2}) \\
    &= (\frac{vs^2+(\theta-\hat{\theta})'X'X(\theta-\hat{\theta})}{2})^{-\frac{n}{2}}\int_0^\infty t^{\frac{n-2}{2}}e^{-t}dt \quad (t=\frac{vs^2+(\theta-\hat{\theta})'X'X(\theta-\hat{\theta})}{2}\phi) \\
    &= \Gamma(\frac{n}{2})(\frac{vs^2+(\theta-\hat{\theta})'X'X(\theta-\hat{\theta})}{2})^{-\frac{n}{2}} \\
    &= \Gamma(\frac{n}{2})(\frac{vs^2}{2})^{-\frac{n}{2}}(1+\frac{(\theta-\hat{\theta})'X'X(\theta-\hat{\theta})}{vs^2})^{-\frac{n}{2}} \\
\end{align}\] Since the density of the multivariate t distribution
follows the form \[\begin{align}
    \frac{\Gamma((v+d)/2)}{\Gamma(v/2)v^{d/2}\pi^{d/2}|\Sigma|^{1/2}}(1+\frac{(x-\mu)'\Sigma^{-1}(x-\mu)}{v})^{-n/2}
\end{align}\] Here, \(\Sigma^{-1}=\frac{X'X}{s^2}\),
\(\sqrt{\pi}=\Gamma(\frac{1}{2})\), \(x=\theta\) and
\(\mu=\hat{\theta}\).\\
Therefore, the above marginal distribution is \[\begin{align}
    p(\theta|Y) &= \frac{\Gamma(\frac{n}{2})|X'X|^{1/2}s^{-d}}{(\Gamma(1/2))^d\Gamma(v/2)v^{d/2}}(1+\frac{(\theta-\hat{\theta})'X'X(\theta-\hat{\theta})}{vs^2})^{-\frac{n}{2}}
\end{align}\] which is said to be the multivariate t distribution.

    \hypertarget{section}{%
\section{2.13}\label{section}}

    \(\textbf{Solution 1}\)\\
From result 2.1.1,
\(p(\theta, \sigma^2|Y) = p(\sigma^2|s^2)\times p(\theta|\hat{\theta}, \sigma^2)\)
where the marginal distribution of \(\sigma^2\) is \(vs^2\chi^{-2}(v)\)
and the conditional distribution \(\theta\), given \(\sigma^2\), is
\(N(\hat{\theta}, \sigma^2(X'X)^{-1})\).\\
Therefore, \[\begin{align}
    \frac{(\theta-\hat{\theta})'X'X(\theta-\hat{\theta})}{ds^2} &= \frac{(\theta-\hat{\theta})'X'X(\theta-\hat{\theta})/\sigma^2d}{s^2/\sigma^2} \\
    &= \frac{\chi^2(d)/d}{s^2\chi^2(v)/s^2v} \\
    &= \frac{\chi^2(d)/d}{\chi^2(v)/v} \\
    &= F(d,v)
\end{align}\] \(\textbf{Solution 2}\)\\
(From wikipedia)The definition of multivariate t distribution, for the
case of p dimensions is, based on the observation that if \(y\) and
\(u\) are independent and distributed as \(N(0,\Sigma)\) and
\(\chi^2(v)\) respectively, and \(y/\sqrt{u/v}=x-\mu\), then \(x\) has
the density \[\begin{align}
    \frac{\Gamma((v+p)/2)}{\Gamma(v/2)v^{p/2}\pi^{p/2}|\Sigma|^{1/2}}(1+\frac{(x-\mu)'\Sigma^{-1}(x-\mu)}{v})^{-(v+p)/2}
\end{align}\] and is said to be distributed as a multivariate t
distribution with parameters \(\Sigma\), \(\mu\), \(v\).\\
So \[\begin{align}
    \frac{(x-\mu)'\Sigma^{-1}(x-\mu)}{p}&= \frac{v}{u}\frac{y'\Sigma^{-1}y}{p} \\
    &= \frac{y'\Sigma^{-1}y/p}{u/v} \\
    &= \frac{\chi^2(p)}{\chi^2(v)}\\
    &= F(p,v)
\end{align}\] Here, \(\Sigma^{-1}=\frac{X'X}{s^2}\), \(p=d\),
\(x=\theta\) and \(\mu=\hat{\theta}\), hence \[\begin{align}
    \frac{(\theta-\hat{\theta})'X'X(\theta-\hat{\theta})}{ds^2}\sim F(d,v)
\end{align}\] (Actually the above two solutions almost have the same
meaning.)

    \hypertarget{section}{%
\section{2.14}\label{section}}

    \hypertarget{a}{%
\subsection{a}\label{a}}

    The pdf of \(\pi\) is \[\begin{align}
    f(\pi)=\frac{\pi^{\alpha-1}(1-\pi)^{\beta-1}}{B(\alpha,\beta)}
\end{align}\] Since \(\pi=\frac{\lambda\alpha}{\lambda\alpha+\beta}\),
the pdf of \(\lambda\) is \[\begin{align}
    g(\lambda)&= \frac{(\frac{\lambda\alpha}{\lambda\alpha+\beta})^{\alpha-1}(\frac{\beta}{\lambda\alpha+\beta})^{\beta-1}}{B(\alpha,\beta)}\frac{\alpha\beta}{(\lambda\alpha+\beta)^2} \\
    &= \frac{\lambda^{\alpha-1}\alpha^{\alpha}\beta^{\beta}}{(\lambda\alpha+\beta)^{\alpha+\beta}B(\alpha,\beta)} \sim F(2\alpha,2\beta)
\end{align}\] That is, \(\lambda\) has the F distribution.

    \hypertarget{b}{%
\subsection{b}\label{b}}

    Since \(\lambda=e^{2\delta}\), the pdf of \(\delta\) is \[\begin{align}
    p(\delta)&=\frac{e^{2\delta(\alpha-1)}\alpha^{\alpha}\beta^{\beta}}{B(\alpha, \beta)(e^{2\delta}\alpha+\beta)^{\alpha+\beta}}2e^{2\delta} \\
    &= \frac{2(2\alpha)^{\alpha}(2\beta)^{\beta}e^{2\delta\alpha}}{B(\alpha,\beta)(2\beta+2\alpha e^{2\delta})^{\alpha+\beta}}
\end{align}\]

    \hypertarget{c}{%
\subsection{c}\label{c}}

    Let \[\begin{align}
    P(\delta)&=p(\delta)d\delta \\
    &= \frac{2(2\alpha)^{\alpha}(2\beta)^{\beta}e^{2\delta\alpha}}{B(\alpha,\beta)(2\beta+2\alpha e^{2\delta})^{\alpha+\beta}}d\delta
\end{align}\] when \(\alpha\) and \(\beta\) are large, we can use
Stirling's approximation of the Beta function \[\begin{align}
    P(\delta)&= \frac{2(2\alpha)^{\alpha}(2\beta)^{\beta}(\alpha+\beta)^{\alpha+\beta-1/2}}{\sqrt{2\pi}(2\beta+2\alpha e^{2\delta})^{\alpha+\beta}\alpha^{\alpha-1/2}\beta^{\beta-1/2}}e^{2\alpha\delta}d\delta \\
    &= \frac{(2\alpha+2\beta)^{\alpha+\beta}}{\sqrt{2\pi}(2\beta+2\alpha e^{2\delta})^{\alpha+\beta}\sqrt{\frac{1}{2}(\frac{1}{2\alpha}+\frac{1}{2\beta})}}e^{2\alpha\delta}d\delta
\end{align}\] let
\(\mu=\frac{1}{2}\log\left[(1-\frac{1}{2\alpha})/(1-\frac{1}{2\beta})\right]\),
\(\sigma^2=\frac{1}{4}\left(\frac{1}{\alpha}+\frac{1}{\beta}\right)\),
since we want to prove they are approximately the mean and variance of
\(p(\delta)\), we only need to prove that
\(t=\frac{\delta-\mu}{\sigma}\) has the standard normal distribution.\\
It follows \[\begin{align}
    P(\delta)&= \frac{(2\alpha+2\beta)^{\alpha+\beta}}{\sqrt{2\pi}\sigma(2\beta+2\alpha e^{2\delta})^{\alpha+\beta}}e^{2\alpha\delta}d\delta \\
    P(t)&= \frac{1}{\sqrt{2\pi}}\left(\frac{2\alpha+2\beta}{2\alpha e^{2(t\sigma+\mu)}+2\beta}\right)^{\alpha+\beta}e^{2\alpha(t\sigma+\mu)}dt \\
    &= \frac{1}{\sqrt{2\pi}}\left(\frac{2\alpha+2\beta}{2\alpha e^{2\beta(t\sigma+\mu)/(\alpha+\beta)}+2\beta e^{-2\alpha(t\sigma+\mu)/(\alpha+\beta)}}\right)^{\alpha+\beta} dt
\end{align}\] Use Taylor expansion for the denominator, \[\begin{align}
    2\alpha e^{2\beta(t\sigma+\mu)/(\alpha+\beta)}+2\beta e^{-2\alpha(t\sigma+\mu)/(\alpha+\beta)}&= 2\alpha(1+\frac{2\beta(t\sigma+\mu)}{\alpha+\beta}+\frac{2\beta^2(t\sigma+\mu)^2}{(\alpha+\beta)^2} +O(\frac{\beta^3}{(\alpha+\beta)^3})) \\ &+ 2\beta(1-\frac{2\alpha(t\sigma+\mu)}{\alpha+\beta}+\frac{2\alpha^2(t\sigma+\mu)^2}{(\alpha+\beta)^2}+ O(\frac{\alpha^3}{(\alpha+\beta)^3})) \\
    &= 2\alpha+2\beta+\frac{4\alpha\beta(t\sigma+\mu)^2}{\alpha+\beta}
\end{align}\] So \(P(t)\) becomes \[\begin{align}
    P(t)&= \frac{1}{\sqrt{2\pi}}(1+\frac{2\alpha\beta(t\sigma+\mu)^2}{(\alpha+\beta)^2})^{-(\alpha+\beta)} dt \\
\end{align}\] Use logarithms \[\begin{align}
    -(\alpha+\beta)\log\left(1+\frac{2\alpha\beta(t\sigma+\mu)^2}{(\alpha+\beta)^2}\right) &= -(\alpha+\beta)\left(\frac{2\alpha\beta(t\sigma+\mu)^2}{(\alpha+\beta)^2}-\frac{1}{2}\left(\frac{2\alpha\beta(t\sigma+\mu)^2}{(\alpha+\beta)^2}\right)^2 + \\ \sum_{r=3}^\infty \frac{(-1)^{r+1}}{r}\left(\frac{2\alpha\beta(t\sigma+\mu)^2}{(\alpha+\beta)^2}\right)^r\right) \\
    &= -\frac{t^2\sigma^2+2t\sigma\mu+\mu^2}{2\sigma^2} + \frac{2\alpha^2\beta^2(t\sigma+\mu)^4}{(\alpha+\beta)^3}+\sum_{r=3}^\infty (-1)^r\frac{[2\alpha\beta(t\sigma+\mu)^2]^r}{r(\alpha+\beta)^{2r-1}} \\
    &= -\frac{t^2}{2}-\frac{t\mu}{\sigma}-\frac{\mu^2}{2\sigma^2}+\frac{2\alpha^2\beta^2}{(\alpha+\beta)^2}\frac{(t\sigma+\mu)^4}{\alpha+\beta} + \sum_{r=3}^\infty (-1)^r\frac{[2\alpha\beta(t\sigma+\mu)^2]^r}{r(\alpha+\beta)^{2r-1}}
\end{align}\] Let \(U=\mu/\sigma\), obviously, \[\begin{align}
    \mu=\frac{1}{2}\log\left[(1-\frac{1}{2\alpha})/(1-\frac{1}{2\beta})\right]=\frac{1}{2}\left[\log(1-\frac{1}{2\alpha})-\log(1-\frac{1}{2\beta})\right]<\frac{1}{2}(\frac{1}{2\beta}-\frac{1}{2\alpha}) <\frac{1}{4}(\frac{1}{\beta}+\frac{1}{\alpha})=\sigma^2
\end{align}\] so \(U<\sigma\). Since \(\lim \sigma=0\),\(\lim \mu=0\),
we have \(\lim U=0\) and \(\lim U^2=0\). Consider
\(\frac{\alpha^2+\beta^2(t\sigma+\mu)^4}{(\alpha+\beta)^3}=\frac{(t\sigma+\mu)^4}{16(\alpha+\beta)\sigma^4}=\frac{(t+U)^4}{16(\alpha+\beta)}\),
and \(\lim \frac{(t+U)^4}{16(\alpha+\beta)}=0\). Similarly,
\(\lim \sum_{r=3}^\infty (-1)^r\frac{[2\alpha\beta(t\sigma+\mu)^2]^r}{r(\alpha+\beta)^{2r-1}}=0\).\\
Hence, \[\begin{align}
    P(t)=\frac{1}{\sqrt{2\pi}} \exp(-\frac{t^2}{2})dt
\end{align}\] This completes the proof.\\
Since \(\log(\pi/(1-\pi))=2\delta-\log(\beta/\alpha)\), this is a linear
transformation, so its mean and variance are \[\begin{align}
    &2\times \frac{1}{2}\log\left[(1-\frac{1}{2\alpha})/(1-\frac{1}{2\beta})\right] - \log(\beta/\alpha)=\log\left[(\alpha-\frac{1}{2})/(\beta-\frac{1}{2})\right] \\
    &4\times \frac{1}{4}\left(\frac{1}{\alpha}+\frac{1}{\beta}\right)=\left(\frac{1}{\alpha}+\frac{1}{\beta}\right)
\end{align}\]

    \hypertarget{d}{%
\subsection{d}\label{d}}

    From question c, we can use normal approximation for \(\pi\) and
\(\rho\), so the mean and variance of \(\log(\pi/(1-\pi))\) and
\(\log(\rho/(1-\rho))\) are \[\begin{align}
    \mu_{\pi}&= \log\left[(\alpha_0+x-\frac{1}{2})/(\beta_0+m-x-\frac{1}{2})\right] \\
    \mu_{\rho}&= \log\left[(\gamma_0+y-\frac{1}{2})/(\delta_0+n-y-\frac{1}{2})\right] \\
    \sigma_{\pi}^2&= \left(\frac{1}{\alpha_0+x}+\frac{1}{\beta_0+m-x}\right) \\
    \sigma_{\rho}^2&= \left(\frac{1}{\gamma_0+y}+\frac{1}{\delta_0+n-y}\right)
\end{align}\] Since they are independent, the mean and variance of
\(\log(\pi/(1-\pi)) - \log(\rho/(1-\rho))\) is \[\begin{align}
    \mu_1&= \log\left[\frac{(\alpha_0+x-\frac{1}{2})(\delta_0+n-y-\frac{1}{2})}{(\beta_0+m-x-\frac{1}{2})(\gamma_0+y-\frac{1}{2})}\right] \\
    \sigma_1^2&= \frac{1}{\alpha_0+x}+\frac{1}{\beta_0+m-x}+\frac{1}{\gamma_0+y}+\frac{1}{\delta_0+n-y}
\end{align}\]

    \hypertarget{reference}{%
\section{Reference}\label{reference}}

    {[}1{]}Clyde, M., Rundel, M. C., Rundel, C., Banks, D., Chai, C., \&
Huang, L. (2020). An Introduction to Bayesian Thinking-A Companion to
the Statistics with R Course. GitHub repository: GitHub.\\
{[}2{]}Williamson, P. P. (2009). Bayesian Alternatives to the F Test for
Two Population Variances. Communications in Statistics-Theory and
Methods, 38(14), 2288-2301.\\
{[}3{]}Aroian, L. A. (1941). A study of RA Fisher's z distribution and
the related F distribution. The Annals of Mathematical Statistics,
12(4), 429-448.\\
{[}4{]}Wikipedia contributors. (2021, September 15). Multivariate
t-distribution. In Wikipedia, The Free Encyclopedia. Retrieved 08:38,
October 8, 2021, from
https://en.wikipedia.org/w/index.php?title=Multivariate\_t-distribution\&oldid=1044413090


    % Add a bibliography block to the postdoc
    
    
    
\end{document}
